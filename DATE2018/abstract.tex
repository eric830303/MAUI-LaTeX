\begin{abstract}
\begin{comment}%TVA, DCC??
Device aging, which causes significant loss on circuit performance and lifetime, has been a primary factor in reliability degradation of nanoscale designs. In this paper, we propose to take advantage of clock skews (i.e., make them useful for aging tolerance) by manipulating these time-varying skews to compensate for the performance degradation of logic networks. We propose two methodologies of clock skew manipulation. First one is based on V\textsubscript{th} assignment. The second one is to manipulate the aging behavior of clock branches. The goal is to assign achievable/reasonable clock skews in a circuit by the two approaches, such that its overall performance degradation due to aging can be minimized, that is, the lifespan can be maximized. On average, 25\% aging tolerance can be achieved with insignificant design overhead.
\end{comment}
%DATE 2018 Abstract
Device aging, which causes significant loss on circuit performance and lifetime, has been a primary factor in reliability degradation of nanoscale designs. In this paper, we propose to take advantage of aging-induced clock skews (i.e., make them useful for aging tolerance) by manipulating these time-varying skews to compensate for the performance degradation of logic networks. The goal is to assign achievable/reasonable aging-induced clock skews in a circuit, such that its overall performance degradation due to aging can be minimized, that is, the lifespan can be maximized.  On average, 25\% aging tolerance can be achieved with insignificant design overhead. Moreover, we also apply V\textsubscript{th} assignment to further mitigate the aging-induced degradation of logic networks. Averagely, 39.74\% aging tolerance can be achieved when aging manipulation and V\textsubscript{th} assignment are applied.
\end{abstract}