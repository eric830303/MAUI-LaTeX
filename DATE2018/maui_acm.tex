 

\documentclass[format=acmsmall, review=false, screen=true]{acmart}

\usepackage[utf8]{inputenc}
\usepackage[english]{babel}

% Metadata Information
\acmJournal{TODAES}
%\acmVolume{9}
%\acmNumber{4}
\acmArticle{0}
\acmYear{2018}
\acmMonth{0}
\copyrightyear{2018}
%\acmArticleSeq{9}

% Copyright
%\setcopyright{acmcopyright}
\setcopyright{acmlicensed}
%\setcopyright{rightsretained}
%\setcopyright{usgov}
%\setcopyright{usgovmixed}
%\setcopyright{cagov}
%\setcopyright{cagovmixed}

%--------- DOI ---------------------
%\acmDOI{0000001.0000001}


\usepackage{graphicx}
\usepackage{subfigure}
\usepackage{amsmath, amsthm, amssymb}
\usepackage{multirow}
%LaTeX package conflicts with several others over the use of the algorithm identifier.  
%A common indicator is something like this message: Too many }'s.l.1616     }
%To resolve the issues, simply put the following just before the inclusion of the algorithm2
\makeatletter
\newif\if@restonecol
\makeatother
\let\algorithm\relax
\let\endalgorithm\relax
\usepackage[linesnumbered,boxed,ruled]{algorithm2e}
\SetAlFnt{\small}
\SetAlCapFnt{\small}
\SetAlCapNameFnt{\small}
\SetAlCapHSkip{0pt}
\IncMargin{-\parindent}



\usepackage{csquotes}
\usepackage{indentfirst}
\usepackage{enumitem}
%\usepackage[style=acm,backend=bibtex8,maxnames=3,minnames=1,sorting=none]{biblatex}
%\usepackage{natbib}
%\bibliographystyle{unsrtnat}
\usepackage[version=4]{mhchem}
\usepackage[bottom]{footmisc}
\usepackage[normalem]{ulem}


\usepackage{afterpage}
\usepackage{comment}
% graphics path
\graphicspath{{figures/}}
% figure name bold
\renewcommand{\figurename}{\textbf{Figure}}
% table name bold
\renewcommand{\tablename}{\textbf{Table}}
% footnote mark
\renewcommand{\thefootnote}{\fnsymbol{footnote}}
% add bib resource
%\addbibresource{maui.bib} 
% class format

\newtheorem{class}{Class}
%bibtex font size
\renewcommand*{\bibfont}{\footnotesize}

% correct bad hyphenation here
\hyphenation{optical networks semiconductor}

\pagenumbering{gobble}
%\usepackage{caption}
%\usepackage[font=small,labelfont=bf]{caption}


\received{0 2018}
%\received[revised]{March 2009}
%\received[accepted]{June 2009}






\begin{document}
\title[MAUI: Making Aging Useful, Intentionally]{MAUI: Making Aging Useful, Intentionally}


\author{Tien-Hung Tseng}
\affiliation{%
  \institution{National Chiao Tung University}
  \city{Hsinchu}
  \country{Taiwan}}
\email{eric830303.cs05g@g2.nctu.edu.tw}


\author{Shou-Chun Li}
\affiliation{%
  \institution{National Chiao Tung University}
  \city{Hsinchu}
  \country{Taiwan}
}
\email{scli.cs02g@nctu.edu.tw}


\author{Kai-Chiang Wu}
\affiliation{%
  \institution{National Chiao Tung University}
  \streetaddress{30 Shuangqing Rd}
  \city{Hsinchu}
  \country{Taiwan}
}
\email{kcw@cs.nctu.edu.tw}



\begin{abstract}
Device aging, which causes significant loss on circuit performance and lifetime, has been a primary factor in reliability degradation of nanoscale designs. In this paper, we propose to take advantage of aging-induced clock skews (i.e., make them useful for aging tolerance) by manipulating these time-varying skews to compensate for the performance degradation of logic networks. The goal is to assign achievable/reasonable aging-induced clock skews in a circuit, such that its overall performance degradation due to aging can be minimized, that is, the lifespan can be maximized.  On average, 24.95\% aging tolerance can be achieved with insignificant design overhead. Moreover, we employ $V_{th}$ assignment to further tolerate the aging-induced degradation of logic networks. When $V_{th}$ assignment is applied on top of aforementioned aging manipulation, the average aging tolerance can be enhanced to 37.61\%.
\end{abstract}


\begin{CCSXML}
<ccs2012>
<concept>
<concept_id>10010583.10010682</concept_id>
<concept_desc>Hardware~Electronic design automation</concept_desc>
<concept_significance>500</concept_significance>
</concept>
<concept>
<concept_id>10010583.10010682.10010696</concept_id>
<concept_desc>Hardware~Modeling and parameter extraction</concept_desc>
<concept_significance>500</concept_significance>
</concept>
<concept>
<concept_id>10010583.10010750</concept_id>
<concept_desc>Hardware~Robustness</concept_desc>
<concept_significance>500</concept_significance>
</concept>
<concept>
<concept_id>10010583.10010750.10010762.10010763</concept_id>
<concept_desc>Hardware~Aging of circuits and systems</concept_desc>
<concept_significance>500</concept_significance>
</concept>
</ccs2012>
\end{CCSXML}

\ccsdesc[500]{Hardware~Electronic design automation}
\ccsdesc[500]{Hardware~Modeling and parameter extraction}
\ccsdesc[500]{Hardware~Robustness}
\ccsdesc[500]{Hardware~Aging of circuits and systems}
\keywords{Clock network, Aging, Degradation, Reliability}





\maketitle

%\input abstract
\input introduction
\input related
\input motivate
\input preliminary
\input framework
\input exp
\input conclude



\bibliographystyle{ACM-Reference-Format}
\bibliography{maui}


% that's all folks
\end{document}


