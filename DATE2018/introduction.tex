\section{INTRODUCTION}
%1
The design and manufacturing of semiconductor devices have recently experienced dramatic innovations, at the cost of downgrading reliability of nanoscale integrated circuits (ICs) and system-on-chips (SOCs). The 2013 ITRS~\cite{itrs2013} projects that the long-term reliability of sub-100nm technology nodes can reach a noteworthy order of $10^3$ FITs (failures in $10^9$ hours). Soft errors, thermal and aging effects are some of the major challenges driving reliability-aware IC/SOC design techniques. With the continuous scaling of transistor dimensions, device aging, which causes temporal performance degradation and potential wear-out failure, is becoming increasingly dominant for lifetime reliability concerns because of limited timing margins~\cite{mcpherson2006reliability}. Therefore, the need of a verification and optimization flow considering aging effects emerges as a key factor in guaranteeing reliable and sustainable operation over a required lifespan.
%2
\textit{Bias temperature instability} (BTI) is known for prevailing over other device aging phenomena, in terms of dependence on the scaling of nanometer technologies. BTI~\cite{schroder2003negative} is a MOSFET aging phenomenon that occurs when transistors are stressed under bias (positive or negative, i.e., $V_{gs} = \pm V_{dd}$) at elevated temperature. As a result of the dissociation of Si-H bonds along the \ce{{Si-SiO}_2} interface, BTI-induced MOSFET aging manifests itself as an increase in the threshold voltage ($V_{th}$) and decrease in the drive current ($I_{ds}$)~\cite{stathis2006negative}, which in turn lengthen the propagation delays of logic gates/paths. Experiments on MOSFET aging~\cite{chakravarthi2004comprehensive} indicate that BTI effects grow exponentially with higher operating temperature and thinner gate oxide. If the thickness of gate oxide shrinks down to 4nm, the circuit performance can be degraded by as much as 15\% after 10 years of stress and lifetime will be dominated by BTI~\cite{kimizuka1999impact}. In contrast, when the stress condition is relaxed ($V_{gs}$ = 0), the aging mechanism can be recovered partially~\cite{kumar2006analytical} and the threshold voltage decreases toward the nominal value by more than 75\% if the recovery phase lasts sufficiently long~\cite{wang2010impact}.
%3
In addition to the aging effect on the logic gates/paths of a circuit, the impact of aging on the clock network should not be ignored. Considering both \enquote{the aging of logic networks} and \enquote{the aging of clock networks} is essential since unbalanced aging of clock networks can greatly affect circuit performance by inducing clock skews, as a result of non-uniform increases in clock latency from the clock source to different terminals. Prior work on addressing such clock skews due to aging (i.e., aging-induced clock skews) mainly attempts to balance the aging effects on various clock sub-networks, such that aging-induced clock skews can be minimized~\cite{chen2013novel, huang2013low, chakraborty2013skew}. However, suppressing aging-induced clock skews may be difficult and costly, especially for clock-gated designs where the rate of aging varies from one clock sub-network to another~\cite{lai2014bti}.

%4
Then think about the following question: if one has to work hard on (and incur significant overhead for) minimizing aging-induced clock skews but it turns out that there still exist stubborn non-zero skews, why don't we try to take advantage of them? We do; we propose to intentionally make aging-induced clock skews useful for aging tolerance. More specifically, we compensate for the performance degradation (due to aging) of logic networks by exploring \enquote{useful} clock skews, based on the concept of time borrowing. There are two key idea in our framework: (1) The first idea is to manipulate the rates of aging on different clock branches. Note that the rate of aging depends on the stress time, defined as the amount of time during which a PMOS/NMOS transistor is stressed under negative/positive bias. For clock drivers (comprising pairs of inverters), their stress times are proportional to the duty cycle of the clock signal. Therefore, our aging manipulation on clock branches is implemented by changing the duty cycle of a clock waveform delivered to each of the clock branches. (2) The second idea is based on V\textsubscript{th} assignment. That is, the threshold voltages of  certain clock buffers are replaced with higher value, such that the delay of the clock buffers are lengthened. The two idea are utilized together to implement the concept of timing borrowing in the designs, such that aging-induced performance degradation can be mitigated effectively, with little design penalty.
%MAUI DATE 2018
%4
\begin{comment}
Then think about the following question: if one has to work hard on (and incur significant overhead for) minimizing aging-induced clock skews but it turns out that there still exist stubborn non-zero skews, why don't we try to take advantage of them? We do; we propose to intentionally make aging-induced clock skews useful for aging tolerance. More specifically, we compensate for the performance degradation (due to aging) of logic networks by exploring \enquote{useful} aging-induced clock skews, based on the concept of time borrowing. Note that the rate of aging depends on the stress time, defined as the amount of time during which a PMOS/NMOS transistor is stressed under negative/positive bias. For clock drivers (comprising pairs of inverters), their stress times are proportional to the duty cycle of the clock signal. The key idea of our framework is to manipulate the rates of aging on different clock branches, by changing the duty cycle of a clock waveform delivered to each of the clock branches. The proposed framework succeeds in mitigating effective aging-induced performance degradation with little design penalty.
\end{comment}