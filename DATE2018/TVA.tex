 \section{FURTHER OPTIMIZATION BY V\textsubscript{TH} ASSIGNMENT}
\label{sec:TVA}
% (1)
In the section, the technique of V\textsubscript{th} assignment is involved in the proposed framework to manipulate clock skew for aging tolerance. The idea comes from the aforementioned prior work~\cite{chen2013novel}, which exploits the technique to minimize the aging-induced clock skew, while it does not make the skew useful. Since V\textsubscript{th} assignment can also manipulate the clock skew, why do not we use the technique to achieve aging tolerance? We do. The two techniques,  V\textsubscript{th} assignment and aging manipulation, are applied together to further optimize the required $T_c$. In other words, in addition to aging manipulation based on DCC insertion/deployment, we assign the threshold voltages of certain clock buffers, such that the latencies of the buffers are changed, and so are the slacks, which give us the opportunity to further mitigate the aging-induced degradation of logic network based on time borrowing. Note that, we only reassign the threshold voltages of clock buffers rather than the counterparts of logic gates. 

The rest of the section is organized as follows: In Section~\ref{sec:TVA:example}, we use an example to demonstrate the effectiveness of V\textsubscript{th} assignment for aging tolerance and in Section~\ref{sec:TVA:leader} we introduce the rule/mechanism of V\textsubscript{th} assignment, followed by~\ref{sec:TVA:framework}, where we explain how we convert the rule/mechanism into CNF clauses, and  introduces the new timing constraints while V\textsubscript{th} assignment is considered. Finally, in Section~\ref{sec:TVA:experiment}, we show the experimental results, which are optimized by inserting DCC and assigning V\textsubscript{th} of clock buffers.
% (2)
%--------------------------------- Motivating Example -------------------------------------------------------
\subsection{Motivating Example considering V\textsubscript{th} assignment}
\label{sec:TVA:example}
We use the illustrative example to explain how we further improve the aging tolerance by comparing the two examples of Section~\ref{sec:motivate} and this section, based on the design in Figure~\ref{fig:sub:example}. The example in Section~\ref{sec:motivate} only considers the aging manipulation based on DCC deployment/insertion; however, in this section, the two techniques, DCC deployment and V\textsubscript{th} assignment are applied together to optimize/reduce required $T_c$. In this example, we let the DCC deployment same with that in the former example (in Section~\ref{sec:motivate}), i.e., 20\% DCC and 80\% DCC are inserted at the inputs of buffer 1 and buffer 2, respectively. Moreover, we begin assigning high V\textsubscript{th} to the certain clock buffers, which are in the intervals from buffer 2 to \ce{FF_x} and  to \ce{FF_y}. In other words, buffer 2 and its downstream buffers are assigned to high V\textsubscript{th}. To include the aging rates of high-V\textsubscript{th} buffers in the setup-time constraint, we assume that, the intrinsic delay of high-V\textsubscript{th} buffer is 1.2X longer than that of nominal buffer (whose V\textsubscript{th} are not reassigned), and aging rates of high-V\textsubscript{th} buffer, with the duty cycle of 20\%, 40\%, 50\% and 80\%, are 0.5\%, 4.1\%, 5.4\% and 8.2\%, respectively. Note that, the aging rates of high-V\textsubscript{th} buffer are lower than those of nominal buffer, because the higher/lower V\textsubscript{th} leads to lower/higher aging rates~\cite{chen2013novel}. Consider the new aging factors in the setup-time constraints on \ce{L_{XY}} and \ce{L_{YZ}}:
\begin{equation}
	\mbox{\fontsize{8}{9.6}\selectfont \ce{L_{XY}}:\quad$\textbf{1.09}C_X+1.2T_{cq}+1.15D_{XY}+1.2T_{su}<\textbf{(1.2+0.08)}C_Y+T_c$} 
	\label{eq:lxy2}
\end{equation}
\begin{equation}
	\centering
	\mbox{\fontsize{8}{9.6}\selectfont \ce{L_{YZ}}:\quad$\textbf{(1.2+0.08)}C_Y+1.2T_{cq}+1.15D_{YZ}+1.2T_{su}<\textbf{(1.2+0.08)}C_Z+T_c$} 
	\label{eq:lyz2}
\end{equation}
By re-arranging Equations (\ref{eq:lxy2}) and (\ref{eq:lyz2}):
\begin{flalign*}
	\hspace{1.2em}\ce{L_{XY}}: T_c &> 96.5 &\\
	\hspace{1.2em}\ce{L_{YZ}}: T_c &> 104
\end{flalign*}

Apparently, the required $T_c$ is further reduced/optimized from 108.5 (in Section~\ref{sec:motivate}) to 104 by inserting two DCCs and V\textsubscript{th} assignment in the existing synthesized clock network. We also observe that, the required $T_c$ is not dominated by \ce{L_{XY}} anymore (in Section~\ref{sec:motivate}); instead, it turns out that \ce{L_{YZ}} dominates $T_c$. As it can be seen, when the two techniques, V\textsubscript{th} assignment and aging manipulation  (by DCC insertion), are applied together, the skew for \ce{L_{XY}}, which equals 1.28$C_Y$ minus 1.09$C_X$, is larger than that in Section~\ref{sec:motivate}. Therefore, the new skew for \ce{L_{XY}} is more useful/beneficial and accounts for the better optimization of required $T_c$. 
Additionally, when it comes to the timing-borrowing mechanism of the two examples, there exists a difference: The timing-borrowing mechanism in Section~\ref{sec:motivate}, is achieved by the aging-induced clock skew, caused by manipulating the duty-cycle delivered to flip-flops. However, in the example, the timing-borrowing mechanism is based on aging-induced clock skew and \textit{tech-induced} clock skew, which is caused by manipulating the technology of clock buffers, i.e., assign the threshold voltages of clock buffers to specific value. 

%--------------------------------- Leader -------------------------------------------------------
\subsection{Technology Leader}
\label{sec:TVA:leader}
When V\textsubscript{th} assignment is considered, how to select shortlisted clock buffers arises as a problem. The so-called shortlisted buffers are those whose V\textsubscript{th} will be assigned (i.e., their technology is changed) to a specific value. Because the possibilities of shortlisted buffers are numerous, we need to reduce the complexity of the problem. Thus, we introduce the rule of V\textsubscript{th} assignment by explaining \textit{technology leader}. 

If the clock buffer is chosen as the \textit{technology leader}, the clock buffer and its downstream buffers are regarded as shortlisted buffers. In other words, the technology of the buffer and the downstream buffers will be replaced with new counterpart. For example, in~\ref{sec:TVA:example}, buffer 2 is actually chosen as the technology leader, so that the shortlisted buffers are buffer 2 and its downstream buffers, which also include buffer 3. Note that, there is a difference between DCC and technology leader. DCC is not a clock buffer but a special-purpose gate, which is inserted at the input of the clock buffer, such that the aging rates of downstream buffers are manipulated; however, technology leader is an existing clock buffer. More specifically, it is selected from the buffers in the existing clock tree and indicates where we begin manipulating the technology of downstream buffers toward flip-flops.

% (4)
%--------------------------------- Framework -------------------------------------------------------
\subsection{Proposed Framework considering V\textsubscript{th} Assignment}
\label{sec:TVA:framework}
\begin{figure}
	\centering
	\includegraphics[width=0.9\columnwidth]{Flow_chart_tva.png}
	\caption{The overall flow of the framework when technology leaders are considered }
	\label{fig:flow:tva}
\end{figure}
The proposed framework, which simultaneously considers aging manipulation (DCC) and V\textsubscript{th} assignment (technology leader), is depicted in Figure~\ref{fig:flow:tva}. Compared with the former framework in Figure~\ref{fig:flow}, the framework is also based on a binary search for the minimum clock period ($T_c$) and SAT formulation, while timing constraints need to consider the technology leader selection. The problem, technology leader selection from clock buffers, is formulated as SAT problem, and so is the problem of DCC insertion/deployment. In the sequel, Section~\ref{sec:TVA:leader_encode} explains how the problem of DCC insertion and technology leader selection is encoded by Boolean variables. Section~\ref{sec:TVA:constraints} reviews DCC constraints and introduces technology leader constriants. Section~\ref{sec:TVA:timingconstraint} introduces the timing constraints which consider DCC deployment and technology leader selection.

%--------------------------------- Encoding -------------------------------------------------------
% (5)
\subsubsection{Encoding for DCC and Technology Leader Deployment}
\label{sec:TVA:leader_encode}
The problem of DCC deployment and technology leader selection needs to be encoded into Boolean representation before being transformed into a SAT-based formulation. Assume that a total of $N$ types of DCCs can be chosen. Including the DCC-free case where no DCC is inserted, there are ($N$ + 1) possibilities of DCC insertion for each clock buffer. Furthermore, we also assume that a total of $M$ types of technology leaders can be chosen. Note that, each type of technology leader represents an individual technology. Including the nominal technology, there are ($M$ + 1) possibilities of technology leader selection for each clock buffer. We denote a clock buffer by $p\left(1 \leq p \leq P\right)$ where $P$ is the total count of clock buffers and $p$ is buffer index. For each clock buffer, there exist two types of Boolean variables, $B_{p,q}$ and $B_{p,r}$ ($1 \leq q \leq Q < r \leq R$, $Q = \lceil \lg (N + 1)\rceil$ and $R = \lceil \lg \{(N + 1)(M + 1\}\rceil$), where $\left\{B_{p,1}, B_{p,2},\dotsc, B_{p,Q}\right\}$ encode the aforementioned ($N$ + 1) possibilities of DCC insertion at the input of buffer $p$ and $\left\{B_{p,Q+1}, B_{p,Q+2},\dotsc, B_{p,R}\right\}$ encode the ($M$ + 1) possibilities of leader selection of buffer $p$.
%The problem of DCC deployment and technology leader selection needs to be encoded into Boolean representation before being transformed into a SAT-based formulation. Assume that a total of $N$ types of DCCs can be chosen. Including the DCC-free case where no DCC is inserted, there are ($N$ + 1) possibilities of DCC insertion for each clock buffer. Furthermore, we also assume that a total of $M$ types of technology leaders can be chosen. Note that, each type of technology leader represents an individual technology. Including the nominal technology, there are ($M$ + 1) possibilities of technology leader selection for each clock buffer. We denote a clock buffer by $p\left(1 \leq p \leq P\right)$ where $P$ is the total count of clock buffers. For each clock buffer, there exist two types of Boolean variables, $B_{p,q}$ and $B_{p,r}$, where $1 \leq p \leq P$, $1 \leq q \leq Q \leq r \leq R$, $Q = \lceil \lg (N + 1)\rceil$ and $R = \lceil \lg \{(N + 1)(M + 1\}\rceil$. $B_{p,q}$ are used to encode the ($N$ + 1) possibilities of DCC insertion, and $B_{p,r}$ are used to encode the ($M$ + 1) possibilities of technology leader selection.


\begin{figure}
    \centering
    \includegraphics[width=0.9\columnwidth]{example_of_dcc_leader.png}
    \caption{An example with DCC deployment and technology selection}
    \label{fig:example_dcc_tva}
\end{figure}

Without loss of generality, we assume $N$ = 3, and $M$ = 1. Thus, there are three types of DCCs, which are assumed to be 20\%, 40\%, and 80\% DCCs, as shown in Figure~\ref{fig:dcctype}. In addition, there is one type of technology leader, which is assumed to be high-V\textsubscript{th} leader. Note that, if the clock buffer is selected as the high-V\textsubscript{th} leader, the technology of the buffer and the associated downstream ones is replaced with high-V\textsubscript{th} counterpart. Since we assume three types of DCC and one type of technology leader, three Boolean variables are used for encoding eight possibilities of DCC and technology leader at any buffer. The eight possibilities can be encoded as follows:

{\small
\begin{tabular}{  c  c  c  c  }
  	 & Leader type & DCC type & $\{B_{p,3}, B_{p,2}, B_{p,1}\}$ \\ 
  	(1)\quad & Nominal V\textsubscript{th} & None & \{0, 0, 0\} \\ 
  	(2)\quad & Nominal V\textsubscript{th} &20\% &  \{0, 0, 1\} \\ 
  	(3)\quad & Nominal V\textsubscript{th} &40\% &  \{0, 1, 0\} \\ 
  	(4)\quad & Nominal V\textsubscript{th} &80\% &  \{0, 1, 1\} \\ 
	(5)\quad & high-V\textsubscript{th} & None & \{1, 0, 0\} \\ 
  	(6)\quad & high-V\textsubscript{th} & 20\% &  \{1, 0, 1\} \\ 
  	(7)\quad & high-V\textsubscript{th} & 40\% &  \{1, 1, 0\} \\ 
  	(8)\quad & high-V\textsubscript{th} & 80\% &  \{1, 1, 1\} \\ 
\end{tabular}}


For example, in Figure~\ref{fig:example_dcc_tva}, the DCC deployment is same with that in Figure~\ref{fig:sub:dcci2} (i.e., 20\% DCC at buffer 3 and 80\% DCC at buffer 6), but the buffer 5 is selected as the high-V\textsubscript{th} leader, which is denoted by a black flag. Thus, the technology of buffer 5, 6 and 7 is replaced with the high-V\textsubscript{th} counterpart. Therefore, {\small $\left\{B_{3,3}, B_{3,2}, B_{3,1}\right\}$ = \{0, 0, 1\}}, {\small $\left\{B_{5,3}, B_{5,2}, B_{5,1}\right\}$ = \{1, 0, 0\}}, {\small $\left\{B_{6,3}, B_{6,2}, B_{6,1}\right\}$ = \{0, 1, 1\}}, and {\small $\left\{B_{p,3}, B_{p,2}, B_{p,1}\right\}$ = \{0, 0, 0\}} for $p$ = 1, 2, 4 or 7.

As mentioned in Section~\ref{subsec:eddcd}, the 20\% DCC mitigates the aging of buffer 3 and its downstream buffers, but the 80\% DCC aggravates the aging of buffer 6 and its downstream buffers. Moreover, since the buffer 5 is a high-V\textsubscript{th} leader, buffer 5, 6 and 7 are changed as high-V\textsubscript{th} buffers, implying the latency from buffer 5 to buffer 7 is lengthened. This way, the clock skew becomes larger than that in Figure~\ref{fig:sub:dcci2}, so that the required $T_c$ can be further reduced, based on time borrowing. Note that, as we mentioned in Section~\ref{sec:TVA:example}, the $T_c$ reduction in Figure~\ref{fig:sub:dcci2} is only due to the useful aging-induced clock skew; however, the  further $T_c$ reduction in Figure~\ref{fig:example_dcc_tva} is both based on useful aging-induced and tech-induced clock skew, which is caused by the V\textsubscript{th} assignment of clock buffers. 
% (6)

%--------------------------------- Constraints - Preface -------------------------------------------------------
\subsubsection{Technology Leader Constraints and DCC Constraints}
\label{sec:TVA:constraints}
\begin{figure}
    \centering
    \includegraphics[width=1\columnwidth]{All_types_of_DCCs_and_leaders.png}
    \caption{Generalized DCC insertion and technology leader selection for a target pair of flip-flops}
    \label{fig:g_dcc_leader}
\end{figure}

Figure~\ref{fig:g_dcc_leader} shows a generalized example of DCC insertion and technology leader selection for a pair of flip-flops (\ce{FF_i} and \ce{FF_j}), where there exist aging-critical paths from \ce{FF_i} to \ce{FF_j}. As described in Section~\ref{subsec:dccccc}, a path is defined as an aging-critical path if it is possible to determine the clock period of the circuit, in the presence of aging. Each pair of flip-flops between which there exist aging-critical paths needs to be considered. Here, we use the generalized example to illustrate our SAT-based formulation. The generalized example in Figure~\ref{fig:g_dcc_leader} is similar to that in Figure~\ref{fig:dcctype}. In contrast, we include technology leader selection for each clock buffer in Figure~\ref{fig:g_dcc_leader}. Moreover, we can find that, if the clock buffer, which is selected as a technology leader, is deep in the clock tree (i.e., close to flip-flops), the count of V\textsubscript{th}-reassigned buffers decreases thus the impact of tech-induced clock skew becomes insignificant. The above phenomenon is similar to that in Section ~\ref{subsec:eddcd}, which observes that deeper DCC deployment causes less aging-induced clock skew. Thus, we set a rule of avoiding selecting/inserting leaders/DCCs at a clock tree level, which is larger/deeper than a specified boundary.  The rule greatly reduce the complexity of SAT-based formulation, because a significant fraction of buffers are not considered being inserted DCC at their inputs and being selected as a technology leader. For instance, in Figure~\ref{fig:g_dcc_leader}, dashed buffers and their downstream buffers are not considered. 

In Figure~\ref{fig:g_dcc_leader}, buffers 1 - 7 are candidate locations for DCC insertion and leader selection. According to the encoding mechanism explained in Section~\ref{sec:TVA:leader_encode}, one Boolean variable is introduced to encode the two possibilities of leader selection, for each of the seven buffers. Thus, there are totally 128 (=$2^7$) possibilities of leader selection, only for this pair of flip-flops. In contrast, there is a total of 16,384 (= $4^7$) possibilities of DCC  deployment. If we combine the two techniques (i.e., DCC insertion and leader selection) together, there is totally 2,097,152 possibilities. The total count of possibilities can be obtained by multiplying the possibilities of the two techniques (i.e., $4^7*2^7$ = 2,097,152), or can be explained by the eight possibilities of DCC insertion and leader selection, for each of the seven buffers (i.e., $8^7$ = 2,097,152). Apparently, this make SAT-based formulation tricky because of clause explosion. Therefore, we set the following constraints on DCC insertion and technology leader selection.

%--------------------------------- DCC Constraints -------------------------------------------------------
% (7)
\paragraph{DCC Constraints}
\label{sec:TVA:dcc_c}
The DCC constraints are identical to those in Section~\ref{subsec:dccccc}. That is, at most one DCC exists along a single clock path, from clock source to one of flip-flops. The corresponding clauses can be referred to the 48 clauses in Section~\ref{subsec:dccccc}. It is worth reminding that, after applying DCC constraints, the total possibilities of DCC insertion can be reduced from 16,384 (= $4^7$) to 103. 

%--------------------------------- Tech Leader Constraints -----------------------------------------------
% (8)
\paragraph{Technology Leader Constraints and Corresponding Clauses}
\label{sec:TVA:leader_c}
The leader constraints are similar to DCC constraints. They are defined as follow: At most one technology leader exists along a single clock path, from clock source to one of the flip-flops. To ensure that no more than one leader along any clock path, we use the Boolean variables $B_{p,r}$ to encode leader selection of each buffer. It is worth reminding that $B_{p,r}$ ($Q < r \leq R$) or $\left\{B_{p,Q+1}, B_{p,Q+2},\dotsc, B_{p,R}\right\}$ encode the ($M$ + 1) possibilities of leader selection of buffer $p$. This way, we can generate some clauses to suppress the occurrence that more than one leader along a clock path. Consider buffer 2 (encoded by {\fontsize{8}{8.4}$\left\{B_{2,3}\right\}$}) and buffer 3 (encoded by {\fontsize{8}{8.4}$\left\{B_{3,3}\right\}$}) in Figure~\ref{fig:g_dcc_leader}. If buffer 2 is a leader (i.e., {\fontsize{8}{8.4}$\{B_{2,3}\} \not\equiv \{0\}$}), then buffer 3 must not be a leader (i.e., {\fontsize{8}{8.4}$\{B_{3,3}\} \equiv \{0, 0\}$}), and vice versa. The constraint can be formally written as:
%The leader constraints are similar to DCC constraints. They are defined as follow: At most one technology leader exists along a single clock path, from clock source to one of the flip-flops. To ensure no more than one leader along any clock path, we can use the Boolean ,variables, which are introduced to encode leader selection, i.e., $B_{p,r}$, where $1 \leq p \leq P$, $\lceil \lg \{(N + 1)\} \rceil = Q \leq r \leq R = \lceil \lg \{(N + 1)(M + 1)\} \rceil$, to generate some clauses which suppress the occurrence of having two leaders along a clock path. Consider buffer 2 (encoded by {\fontsize{8}{8.4}$\left\{B_{2,3}\right\}$}) and buffer 3 (encoded by {\fontsize{8}{8.4}$\left\{B_{3,3}\right\}$}) in Figure~\ref{fig:g_dcc_leader}. If buffer 2 is a leader (i.e., {\fontsize{8}{8.4}$\{B_{2,3}\} \not\equiv \{0\}$}), then buffer 3 must not be a leader (i.e., {\fontsize{8}{8.4}$\{B_{3,3}\} \equiv \{0, 0\}$}), and vice versa. The constraint can be formally written as:
{
\fontsize{8}{8.4}
\begin{gather*}
\left(\{B_{2,3}\} \equiv \{0\}\right) \lor \left(\{B_{3,3}\} \equiv \{0, 0\}\right)
\end{gather*}
}
Next, it can be translated into one CNF clause:
{\fontsize{8}{8.4}($\neg B_{2,3}\lor\neg B_{3,3}$).} 

Any pair of buffers along a single clock path should be constrained in this way. Among buffers 1 $\hyphen$ 7 in Figure~\ref{fig:dcctype}, there are 12 pairs: $\langle1, 2\rangle$, $\langle1, 3\rangle$, $\langle1, 4\rangle$, $\langle1, 5\rangle$, $\langle1, 6\rangle$, $\langle1, 7\rangle$, $\langle2, 3\rangle$, $\langle2, 4\rangle$, $\langle3, 4\rangle$, $\langle5, 6\rangle$, $\langle5, 7\rangle$, $\langle6, 7\rangle$. Each pair translates to one clause and a total of 12 clauses will be generated accordingly.

With leader constraints and corresponding clauses, we can drastically reduce the possibilities of leader selection to be formulated. In the above example where 12 clauses associated with leader constraints are generated, when we only consider the possibilities of leader selection, the possibility count drops from 128 (= $2^7$) to 17. In addition, when we consider the possibilities of DCC deployment and leader selection, the possibility count drops from 2,097,152 to 1751 due to DCC and leader constraints. In the next subsection, we describe what the 17 and 1751 possibilities are and how they are translated into final CNF representation.

%--------------------------------- Timing Constraints (DCC+TVA) - Preface -----------------------------------------------
% (9)
\subsubsection{Timing Constraints considering DCC and Technology Leader}
\label{sec:TVA:timingconstraint}
\begin{figure*}
    \centering
    \subfigure[Case 2-1: One leader on the common clock path (e.g., at buffer 1), and two DCCs]{
    	\label{fig:sub:dccleaderi1}
        \includegraphics[width=0.9\columnwidth]{2DCC_1Leader.png}
    }
    \hspace{1cm}
    \subfigure[Case 2-2: one leader on one of the divergent clock paths, or class 3: two leaders, one on each of the divergent clock paths]{
    	\label{fig:sub:dccleaderi2}
        \includegraphics[width=0.9\columnwidth]{2DCC_2Leader.png}
    }
    \caption{Examples of DCC insertion}
    \label{fig:dccleaderinsert}
\end{figure*}
The timing constraints, considering the two techniques, is extended from those in Section~\ref{subsec:tccc}. As seen in Section~\ref{subsec:tccc}, given a pair of flip-flops, between which there exists one logic path, the timing (i.e., setup-time and hold-time) constraints must be met based on the inequalities Equation~(\ref{eq:tsu}) and~(\ref{eq:th}). The former timing constraints in Section~\ref{subsec:tccc} only consider the impact of DCC deployment on clock latency. Thus, the new timing constraints here must consider the impact of V\textsubscript{th} assignment on clock latency. 

Given one clock period $T_c$ derived by binary search, one aging-critical logic path, and its associated clock network, the former timing constraints are classified into 3 classes, according to the used DCC count (i.e., no, one, and two DCC insertions). The new timing constraints are extended from the 3 classes, each of which are further classified into 3 subclasses, according to the leader count (i.e., the count of buffers which are selected as leaders). Thus, there are totally 9 (= $3*3$) subclasses based on the count of DCC and leader. For brevity, 6 of the 9 subclasses are omitted in the following discussion. We only discuss the other 3 subclasses, whose DCC deployments are identical, but leader counts vary from 0, 1 to 2. Furthermore, due to the aforementioned DCC and leader constraints, the SAT solver will only output a DCC (leader) deployment (selection) where there does not exist more than one DCC (leader) along any clock path. Thus, in the following discussion, the deployment (selection) with more than one DCC (leader) along a single clock path can be ignored. In each subclass, if the DCC (leader) deployment (selection) causes a timing violation within 10 years (i.e., the lifespan specification), then the deployment (selection) will be transformed into CNF clauses, such that the solver will not output the deployment (selection) as results. Here, we explain the generation of CNF clauses by using the example in Figure~\ref{fig:dccleaderinsert}.

%--------------------------------- Timing Constraints (DCC+TVA) - Class 1 -----------------------------------------------
\setcounter{class}{0}
\begin{class}
\label{class:c4}
No buffer on either clock path is selected as a technology leader

Consider the situation that, among buffers 1 $\hyphen$ 7, no buffer is selected as a technology leader. If it causes a timing violation along the aging-critical path within 10 years, then the Boolean representation of the deployment,{\fontsize{8}{8.4}
\begin{gather*}
\left(\{B_{1,3}, B_{1,2}, B_{1,1}\} \equiv \{0, 0, 0\} \right) 
\land \left( \{B_{2,3}, B_{2,2}, B_{2,1}\} \equiv \{0, 0, 0\} \right) \\ \land \dotsb 
\land \left( \{B_{7,3}, B_{7,2}, B_{7,1}\} \equiv \{0, 0, 0\} \right),
\end{gather*}}
equivalent to the following CNF clause:
{\fontsize{8}{8.4}$(B_{1,3} \lor B_{1,2} \lor B_{1,1} \lor B_{2,3} \lor B_{2,2} \lor B_{2,1} \lor \dotsb \lor B_{7,2} \lor B_{7,2} \lor B_{7,1} )$,}
should be generated such that the solver will not output the corresponding DCC deployment and leader selection in the result if the CNF is satisfiable. In this case, a total of 1 CNF clause is generated.
\end{class}

%--------------------------------- Timing Constraints (DCC+TVA) - Class 2 -----------------------------------------------
\begin{class}
\label{class:c5}
Only one buffer is selected as a technology leader

This class can be further classified into 2 sub-classes based on the buffer location of technology leader. \\
\textit{Class 2-1:} Selected buffer (i.e., technology leader) on the common clock path.

In Figure~\ref{fig:dccleaderinsert}, buffer 1 is on the common clock path. Consider the leader selection shown in Figure~\ref{fig:sub:dccleaderi1}: if buffer 1 is the high-V\textsubscript{th} leader and the leader selection causes a timing violation within 10 years, then the Boolean representation of the technology leader selection and the DCC deployment, 
{\fontsize{8}{8.4}
\begin{gather*}
\left(\{B_{1,3}, B_{1,2}, B_{1,1}\} \equiv \{1, 0, 0\} \right) \land \left( \{B_{2,3}, B_{2,2}, B_{2,1}\} \equiv \{0, 0, 0\} \right) \\ \land \left( \{B_{3,3}, B_{3,2}, B_{2,1}\} \equiv \{0, 1, 0\} \right) \land \left( \{B_{4,3}, B_{4,2}, B_{4,1}\} \equiv \{0, 0, 0\} \right) \\ \land \left( \{B_{5,3}, B_{5,2}, B_{5,1}\} \equiv \{0, 0, 0\} \right) \land  \left( \{B_{6,3}, B_{6,2}, B_{6,1}\} \equiv \{0, 1, 1\} \right) \\ \land \left( \{B_{7,3}, B_{7,2}, B_{7,1}\} \equiv \{0, 0, 0\} \right),
\end{gather*}
}
equivalent to the following clause: 
{\fontsize{8}{8.4}$(\neg B_{1,3} \lor B_{1,2} \lor B_{1,1} \lor B_{2,3} \lor B_{2,2} \lor B_{2,1} \lor B_{3,3} \lor \neg B_{3,2} \lor B_{3,1} 
\lor B_{4,3} \lor B_{4,2} \lor B_{4,1} \lor B_{5,3} \lor B_{5,2} \lor B_{5,1} \lor B_{6,3} \lor \neg B_{6,2} \lor B_{6,1}  
\lor B_{7,3} \lor B_{7,2} \lor B_{7,1} )$}, should be generated such that the solver will not output the leader selection and DCC deployment in the result if the CNF is satisfiable. Given that there are 1 choices of technology leader, a total of 1 CNF clause will be generated in the worst case.

%DATE 2018
%if the insertion of a 40\% DCC at buffer 1 causes a timing violation within 10 years, then the Boolean representation of the DCC deployment, $\left(\{B_{1,2}, B_{1,1}\} \equiv \{1, 0\} \right) \land \left( \{B_{2,2}, B_{2,1}\} \equiv \{0, 0\} \right) \land \dotsb \land \left( \{B_{7,2}, B_{7,1}\} \equiv \{0, 0\} \right)$, equivalent to the following clause: $\left(\neg B_{1,2} \lor B_{1,1} \lor B_{2,2} \lor B_{2,1} \lor \dotsb \lor B_{7,2} \lor B_{7,1} \right)$, should be generated such that the solver will not output the deployment in the result if the CNF is satisfiable. Given that there are 3 choices of DCCs, a total of 3 CNF clauses will be generated in the worst case. \\

%\textit{Class 2-2:} \mbox{\fontsize{9}{10.8}\selectfont Selected buffer (i.e., technology leader) is on one of the divergent clock paths}
\textit{Class 2-2:} Selected buffer (i.e., technology leader) is on one of the divergent clock paths.

This class targets buffers 2, 3, 4, 5, 6, 7. If buffer 2 is the high-V\textsubscript{th} leader, and it causes a timing violation within 10 years, then the Boolean representation of the leader selection and the DCC deployment, 
{\fontsize{8}{8.4}
\begin{gather*}
\left(\{B_{2,3}, B_{2,2}, B_{2,1}\} \equiv \{1, 0, 0\} \right) \land \left( \{B_{1,3}, B_{1,2}, B_{1,1}\} \equiv \{0, 0, 0\} \right) \\ 
\land \left( \{B_{3,3}, B_{3,2}, B_{3,1}\} \equiv \{0, 1, 0\} \right) \land \left( \{B_{4,3}, B_{4,2}, B_{4,1}\} \equiv \{0, 0, 0\} \right) \\ 
\land \left( \{B_{5,3}, B_{5,2}, B_{5,1}\} \equiv \{0, 0, 0\} \right) \land \left( \{B_{6,3}, B_{6,2}, B_{6,1}\} \equiv \{0, 1, 1\} \right) \\ 
\land \left( \{B_{7,3}, B_{7,2}, B_{7,1}\} \equiv \{0, 0, 0\} \right),
\end{gather*}
}
equivalent to the following CNF clause: 
{\fontsize{8}{8.4}$(B_{1,3} \lor B_{1,2} \lor B_{1,1} \lor \neg B_{2,3} \lor B_{2,2} \lor B_{2,1} \lor B_{3,3} \lor \neg B_{3,2} \lor B_{3,1} \lor B_{4,3} \lor B_{4,2} \lor B_{4,1} \lor B_{5,3} 
\lor B_{5,2} \lor B_{5,1} \lor B_{6,3} \lor \neg B_{6,2} \lor  \neg B_{6,1} \lor B_{7,3} \lor B_{7,2} \lor B_{7,1} )$}, should be generated such that the solver will not output the leader selection and DCC deployment in the result if the CNF is satisfiable. This class includes 6 candidates: buffers 2, 3, 4, 5, 6, 7, and each has 1 choice of technology leader. Therefore, a total of 6 CNF clauses will be generated in the worst case.
%DATE 2018
%This class targets buffers 2, 3, 4, 5, 6, 7. If the insertion of a 20\% DCC at buffer 3 causes a timing violation within 10 years, then the Boolean representation of the DCC deployment, $\left(\{B_{3,2}, B_{3,1}\} \equiv \{0, 1\} \right) \land \left( \{B_{1,2}, B_{1,1}\} \equiv \{0, 0\} \right) \land \dotsb \land \left( \{B_{7,2}, B_{7,1}\} \equiv \{0, 0\} \right)$, equivalent to the following CNF clause: $\left(B_{3,2} \lor \neg B_{3,1} \lor B_{1,2} \lor B_{1,1} \lor B_{2,2} \lor B_{2,1} \lor \dotsb \lor B_{7,2} \lor B_{7,1} \right)$, should be generated such that the solver will not output the deployment in the result if the CNF is satisfiable. This class includes 6 candidates: buffers 2, 3, 4, 5, 6, 7, and each has 3 choices of DCCs. Therefore, a total of 18 CNF clauses will be generated in the worst case.
\end{class}
%--------------------------------- Timing Constraints (DCC+TVA) - Class 3 -----------------------------------------------
\begin{class}
\label{class:c6}
Two buffers on two divergent clock paths are selected as technology leaders.

Given the DCC deployment and leader selection in Figure~\ref{fig:sub:dccleaderi2} (a 20\% DCC inserted at buffer 3, a 80\% DCC inserted at buffer 6, and buffer 2 and 5 are both  high-V\textsubscript{th} leaders), if it causes a timing violation along the aging-critical path within 10 years, then the Boolean representation of the deployment, 
{\fontsize{8}{8.4}
\begin{gather*}
\left(\{B_{1,3}, B_{1,2}, B_{1,1}\} \equiv \{0, 0, 0\} \right) \land \left( \{B_{2,3}, B_{2,2}, B_{2,1}\} \equiv \{1, 0, 0\} \right) \\ 
\land \left( \{B_{3,3}, B_{3,2}, B_{3,1}\} \equiv \{0, 1, 0\} \right) \land \left( \{B_{4,3}, B_{4,2}, B_{4,1}\} \equiv \{0, 0, 0\} \right) \\ 
\land \left( \{B_{5,3}, B_{5,2}, B_{5,1}\} \equiv \{1, 0, 0\} \right) \land \left( \{B_{6,3}, B_{6,2}, B_{6,1}\} \equiv \{0, 1, 1\} \right) \\ 
\land \left( \{B_{7,3}, B_{7,2}, B_{7,1}\} \equiv \{0, 0, 0\} \right),
\end{gather*}
}
equivalent to the following  CNF clause: 
{\fontsize{8}{8.4}$(B_{1,3} \lor B_{1,2} \lor B_{1,1} \lor \neg B_{2,3} \lor B_{2,2} \lor B_{2,1} \lor B_{3,3} \lor \neg B_{3,2} \lor B_{3,1} \lor B_{4,3} \lor B_{4,2} \lor B_{4,1} \lor \neg B_{5,3} \\
\lor B_{5,2} \lor B_{5,1} \lor B_{6,3} \lor \neg B_{6,2} \lor  \neg B_{6,1} \lor B_{7,3} \lor B_{7,2} \lor B_{7,1} )$}, should be generated such that the solver will not output the deployment in the result if the CNF is satisfiable.

Class 3 considers the selection of the two technology leaders, one among buffers \{2, 3, 4\} and the other one among buffers \{5, 6, 7\}; thus, there are totally 9 (=$3^2$) possibilities of high-V\textsubscript{th} leader selection. Therefore, a total of 9 CNF clauses will be generated in the worst case.

Considering all of the above cases with the given DCC deployment, a maximum number of 1 + 1 + 6 + 9 = 17 clauses can be derived. This is based on the existence of 12 clauses introduced in Section~\ref{sec:TVA:dcc_c}. If we consider the 103 possibilities of DCC deployment in Section~\ref{subsec:tccc}, a total of 1751 (=$103*17$) clauses can be derived in the worst case.

\end{class}

%--------------------------------- Experimental Results ----------------------------------------------------------------
% (10)
\subsection{Experimental Setting and Aging Rate Model for High-V\textsubscript{th} Buffer}
\label{sec:TVA:experiment}
The proposed framework, which simultaneously considers the two techniques, is implemented in C++ and SAT-based formulation is solved by MiniSat. The experimental environment and benchmark circuits are identical to the former setting in Section~\ref{sec:exp}.
Under 10-year aging influence with BTI, the aging rates of clock buffers are obtained from HSPICE. The aging rates of nominal clock buffers with duty cycles of 20\%, 40\%, 50\%, and 80\% are 8.5\%, 12.1\%, 13.5\%, and 16.4\% respectively and the aging rate of logic is obtained by using the same predictive model. 

The timing information (e.g., intrinsic delay and aging rate) of high-V\textsubscript{th} buffers is estimated based on the model proposed in~\cite{andres2016}, which discusses the correlation between BTI and fresh V\textsubscript{th} offsets among transistors. Note that, the V\textsubscript{th} offsets are treated as process variations in~\cite{andres2016}, while here they are regarded as the gaps between high V\textsubscript{th} and nominal counterpart. The correlation is a long-term phenomenon that bridge the V\textsubscript{th} differences among the transistors over a period. Further, a positive/negative V\textsubscript{th} offset leads to a higher/lower fresh V\textsubscript{th}, causing a lower/higher aging speed. Therefore, the gap between high and low V\textsubscript{th} will be gradually converged, letting threshold voltages of transistors, whose fresh ones are different, reach a convergent value. Thus, we can conclude that the gap between high and nominal V\textsubscript{th} will be gradually converged over a specific period. The concept will be used in the latter model to derive the aging rates of high-V\textsubscript{th} buffer. A model in~\cite{andres2016} is proposed to estimate the correlation between fresh V\textsubscript{th} offset and BTI effects:
\begin{equation}
	\centering
	\Delta V_{th\_nbti} = (1 - S_{v} \cdot \Delta V_{th\_offset})  \cdot A \cdot a^n \cdot t^n
	\label{eq:cor}
\end{equation}
\begin{equation}
	\centering
	V_{th} = \Delta V_{th\_nbti} + \Delta V_{th\_offset} + V_{th\_design}
	\label{eq:conv}
\end{equation}
$\Delta V_{th\_offset}$ denotes the fresh V\textsubscript{th} offset, which is the gap between high V\textsubscript{th} and nominal V\textsubscript{th} in the framework. $\Delta V_{th\_nbti}$ denotes the BTI-induced V\textsubscript{th} shift and $V_{th\_design}$ denotes the nominal threshold voltage of the design. $S_{v}$ depends on $\Delta V_{th\_offset}$, and can be derived by following procedures:

\paragraph{Assume the value of $V_{th}$ is convergent}
We assume threshold voltages of all transistors will be convergent after a long period, even if their fresh values are different. In other words, $V_{th}$ is fixed regardless of various $\Delta V_{th\_offset}$, since the aforementioned correlation takes effect. 

\paragraph{Obtain the convergent value of $V_{th}$}
Since $V_{th}$ is fixed regardless of various $\Delta V_{th\_offset}$, we set $\Delta V_{th\_offset}$ to 0 in Equation~(\ref{eq:conv}) to derive the convergent value of $V_{th}$. This way, Equation~(\ref{eq:conv}) can be simplified as Equation~(\ref{eq:conv2}), where the convergent value of $V_{th}$ equals $\Delta V_{th\_nbti}$ plus $V_{th\_design}$. Here, $V_{th\_design}$ is given by technology and $\Delta V_{th\_nbti}$ can be simplified as Equation~(\ref{eq:cor2}) because $V_{th\_offset}$ is set to 0. In Equation~(\ref{eq:conv2}), since $V_{th\_design}$ is known and $\Delta V_{th\_nbti}$ can be derived without unknown $S_{v}$, the convergent value of $V_{th}$ can be obtained.
\begin{equation}
	\centering
	\Delta V_{th\_nbti} = A \cdot a^n \cdot t^n
	\label{eq:cor2}
\end{equation}
\begin{equation}
	\centering
	V_{th} = \Delta V_{th\_nbti} + 0 + V_{th\_design}
	\label{eq:conv2}
\end{equation}
\paragraph{Obtain the value of $S_{v}$ with specific $\Delta V_{th\_offset}$}
Given a specific value of $\Delta V_{th\_offset}$, our objective is to derive corresponding $S_{v}$ value. Since the convergent $V_{th}$ value is obtained in the last step and $V_{th\_design}$ is known, we can derive the corresponding $\Delta V_{th\_nbti}$ using Equation~(\ref{eq:conv}), such that the corresponding $S_{v}$ can be obtained in Equation~(\ref{eq:cor}). 

%Shortly speaking, given a specific value of $\Delta V_{th\_offset}$, our objective is to derive corresponding $S_{v}$ value. We first assume threshold voltages are convergent over a long-term period; then, since all the coefficients of Equation~(\ref{eq:cor}) are known, the value of $\Delta V_{th\_nbti}$ can be obtained.
So far, the conversion from a given specific $\Delta V_{th\_offset}$ to corresponding $\Delta V_{th\_nbti}$ has been constructed. Then, $\Delta V_{th\_nbti}$ must be transformed to aging-induced delay shift. In~\cite{wang2007efficient}, the delay shift is linearly proportional to $\Delta V_{th\_nbti}$:
\begin{equation}
	\centering
	\Delta t_{p\_aged} = C \cdot \Delta V_{th\_nbti}
	\label{eq:vtodelay}
\end{equation}	
where $\Delta t_{p\_aged}$ is BTI-induced delay shift, and $C$ is a constant and fitted to 0.5 after SPICE simulation. Further, the Equation~(\ref{eq:vtodelay}) is modified as following Equation~(\ref{eq:vtodelay2}) to account for the conversion from $\Delta V_{th\_offset}$ to intrinsic delay shift. 
\begin{equation}
	\centering
	\Delta t_{p\_intrinsic} = C \cdot \Delta V_{th\_offset}
	\label{eq:vtodelay2}
\end{equation}	
where $\Delta t_{p\_intrinsic}$ is the delay shift caused by $\Delta V_{th\_offset}$. Up to now, a model is built to convert a given specific $\Delta V_{th\_offset}$ to corresponding $\Delta t_{p\_aged}$ and $\Delta t_{p\_intrinsic}$. As mentioned earlier, $\Delta V_{th\_offset}$ is the fresh offset/gap between high and nominal V\textsubscript{th}. We use the model to derive the intrinsic delay and aging rates of high-V\textsubscript{th} buffer. In our experiments, $\Delta V_{th\_offset}$ is set to a specific value, such that the intrinsic delay of high-V\textsubscript{th} buffers is 1.2X longer than that of nominal buffer, and the aging rates with duty cycle of 20\%, 40\%, 50\%, and 80\% are 0.5\%, 4.1\%, 5.4\%, and 8.2\%, respectively.

\subsection{Experimental Results}
\begin{table*}
\centering
\caption{Experimental results of DCC deployment and high-V\textsubscript{th} assignment}
	\begin{tabular}{l}
	\includegraphics[width=1.7\columnwidth]{Experimental_result_DCC_TVA.png}
	\end{tabular}
\label{table:exp2}
\end{table*}

Table~\ref{table:exp2} shows the experimental results, where Column 4 - 7 demonstrate the former results after applying DCC deployment  and Column 8 - 12 demonstrate the counterparts after applying both DCC deployment and high-V\textsubscript{th} assignment. Column 8 demonstrates the optimum clock period after the two techniques are applied together, denoted by $T_{c\_aged\_opt\_Leader}$. Column 9 demonstrates the used DCC count, Column 10 demonstrates the count of high-V\textsubscript{th} buffers, total count of clock buffers and the ratio of high-V\textsubscript{th} buffers to total clock buffers, and Column 11 demonstrates the run time. The last column demonstrates the improvement, which is the level of aging tolerance and is calculated as:
\begin{gather*}
1 - (T_{c\_aged\_opt\_Leader} - T_{c\_fresh}) / (T_{c\_aged} - T_{c\_fresh})
\end{gather*}
As it can be seen, the proposed framework, which considers the two techniques, can results in lower clock period, implying better improvement for aging tolerance. We can find that, after high-V\textsubscript{th} assignment is included in the former framework, resulting framework may lead to different DCC count. Take \textit{des\_perf} for example, the former DCC count is 7, while the latter DCC count increases to 33. The DCC counts of the two frameworks differ because the DCC deployments are not identical anymore. Specifically speaking, when V\textsubscript{th} assignment is considered, some clock buffers become candidate buffers to be inserted DCC at their inputs, because timing constraints (i.e., setup-time and hold-time) are met based on the inequality Equation (1) and (2), such that DCC can be redeployed to obtain lower Tc. Thus, even thought the two frameworks target the identical benchmark, the DCC deployments/counts may differ. Additionally, the run time of the framework dramatically increases due to numerous possibilities of DCC deployment and leader selection. To be specific, given a pair of flip-flops and associated clock paths, the former framework only considers the possibilities of DCC deployment along the clock paths, while the framework further considers the possibilities of leader selection, for each possibility of DCC deployment. Therefore, the total  possibilities of DCC deployment and leader selection is equal to their mutual multiplication, i.e., DCC possibilities multiplied by leader counterparts, accounting for the dramatic increase of run time.
