\section{CONCLUSION AND FUTURE WORK}
\label{sec:conclude}
In this paper, we propose a novel framework, successfully taking advantage of aging-induced clock skews by inserting limited count of DCCs into the clock tree, to enhance circuit aging tolerance. We transform the problem to a Boolean satisfiability formulation which is then solved by using MiniSat. Experiments demonstrate that our framework gains an average of 25.05\% aging tolerance via inserting at most 15 DCCs. Furthermore, we include the technique of V\textsubscript{th} assignment in the framework, to further improve circuit aging tolerance, by assigning small fraction of clock buffers to high V\textsubscript{th}. The problem of selecting buffers to be assigned high V\textsubscript{th} is also transformed to a Boolean satisfiability formulation, which is solved by MiniSat. Experiments shows that the framework, which considers DCC deployment and V\textsubscript{th} assignment simultaneously, can result in an average of 37.89\% aging tolerance, via inserting at most 33 DCCs and assigning 5.21\% of buffers to high-V\textsubscript{th}.

%------------------- DATE 2018 ------------------------------------------
%In this paper, we propose a novel framework, successfully taking advantage of aging-induced clock skews by inserting limited number of DCCs into the clock tree, to enhance circuit aging tolerance. We transform the problem to a Boolean satisfiability formulation which is then solved by using MiniSat. Experiments demonstrate that our framework gains an average of 25.05\% aging tolerance via inserting at most 15 DCCs and is experimentally verified to be runtime-efficient.

%Process variations (PVs) may shift the delay of each gate, leading to inaccuracy of our optimization results. As indicated in~\cite{wang2010impact}, the transistor with a higher (lower) fresh $V_{th}$ ages at a lower (higher) rate. That is, the variation in $V_{th}$ can be gradually compensated in the aging process. It is demonstrated in~\cite{wang2010impact} that a PV-induced delay shift of 6.2\% can be reduced to 1.6\% after 10-year aging. Even though it is not significant compared to the improvement achieved by our work, we plan to consider the impact of PVs in the future so as to make the proposed idea more robust.