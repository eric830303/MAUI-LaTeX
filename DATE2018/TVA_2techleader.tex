%--------------------------------- Leader -------------------------------------------------------
\subsection{Technology Leader}
\label{sec:TVA:leader}
When V\textsubscript{th} assignment is considered, how to select shortlisted clock buffers arises as a problem. The so-called shortlisted buffers are those whose V\textsubscript{th} will be assigned (i.e., their technology is changed) to a specific value. Because the possibilities of shortlisted buffers are numerous, we need to reduce the complexity of the problem. Thus, we introduce the rule of V\textsubscript{th} assignment by explaining \textit{technology leader}. 

If the clock buffer is chosen as the \textit{technology leader}, the clock buffer and its downstream buffers are regarded as shortlisted buffers. In other words, the technology of the buffer and the downstream buffers will be replaced with new counterpart. For example, in~\ref{sec:TVA:example}, buffer 2 is actually chosen as the technology leader, so that the shortlisted buffers are buffer 2 and its downstream buffers, which also include buffer 3. Note that, there is a difference between DCC and technology leader. DCC is not a clock buffer but a special-purpose gate, which is inserted at the input of the clock buffer, such that the aging rates of downstream buffers are manipulated; however, technology leader is an existing clock buffer. More specifically, it is selected from the buffers in the existing clock tree and indicates where we begin manipulating the technology of downstream buffers toward flip-flops.
