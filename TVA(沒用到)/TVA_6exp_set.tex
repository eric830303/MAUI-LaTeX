 %--------------------------------- Experimental Results ----------------------------------------------------------------
% (10)
\subsection{Experimental Setting and Aging Rate Model for High-V\textsubscript{th} Buffer}
\label{sec:TVA:experiment}
The proposed framework, which simultaneously considers the two techniques, is implemented in C++ and SAT-based formulation is solved by MiniSat. The experimental environment and benchmark circuits are identical to the former setting in Section~\ref{sec:exp}.
Under 10-year aging influence with BTI, the aging rates of clock buffers are obtained from HSPICE. The aging rates of nominal clock buffers with duty cycles of 20\%, 40\%, 50\%, and 80\% are 8.5\%, 12.1\%, 13.5\%, and 16.4\% respectively and the aging rate of logic is obtained by using the same predictive model. 

The timing information (e.g., intrinsic delay and aging rate) of high-V\textsubscript{th} buffers is estimated based on the model proposed in~\cite{andres2016}, which discusses the correlation between BTI and fresh V\textsubscript{th} offsets among transistors. Note that, the V\textsubscript{th} offsets are treated as process variations in~\cite{andres2016}, while here they are regarded as the gaps between high V\textsubscript{th} and nominal counterpart. The correlation is a long-term phenomenon that bridge the V\textsubscript{th} differences among the transistors over a period. Further, a positive/negative V\textsubscript{th} offset leads to a higher/lower fresh V\textsubscript{th}, causing a lower/higher aging speed. Therefore, the gap between high and low V\textsubscript{th} will be gradually converged, letting threshold voltages of transistors, whose fresh ones are different, reach a convergent value. Thus, we can conclude that the gap between high and nominal V\textsubscript{th} will be gradually converged over a specific period. The concept will be used in the latter model to derive the aging rates of high-V\textsubscript{th} buffer. A model in~\cite{andres2016} is proposed to estimate the correlation between fresh V\textsubscript{th} offset and BTI effects:
\begin{equation}
	\centering
	\Delta V_{th\_nbti} = (1 - S_{v} \cdot \Delta V_{th\_offset})  \cdot A \cdot a^n \cdot t^n
	\label{eq:cor}
\end{equation}
\begin{equation}
	\centering
	V_{th} = \Delta V_{th\_nbti} + \Delta V_{th\_offset} + V_{th\_design}
	\label{eq:conv}
\end{equation}
$\Delta V_{th\_offset}$ denotes the fresh V\textsubscript{th} offset, which is the gap between high V\textsubscript{th} and nominal V\textsubscript{th} in the framework. $\Delta V_{th\_nbti}$ denotes the BTI-induced V\textsubscript{th} shift and $V_{th\_design}$ denotes the nominal threshold voltage of the design. $S_{v}$ depends on $\Delta V_{th\_offset}$, and can be derived by following procedures:

\paragraph{Assume the value of $V_{th}$ is convergent}
We assume threshold voltages of all transistors will be convergent after a long period, even if their fresh values are different. In other words, $V_{th}$ is fixed regardless of various $\Delta V_{th\_offset}$, since the aforementioned correlation takes effect. 

\paragraph{Obtain the convergent value of $V_{th}$}
Since $V_{th}$ is fixed regardless of various $\Delta V_{th\_offset}$, we set $\Delta V_{th\_offset}$ to 0 in Equation~(\ref{eq:conv}) to derive the convergent value of $V_{th}$. This way, Equation~(\ref{eq:conv}) can be simplified as Equation~(\ref{eq:conv2}), where the convergent value of $V_{th}$ equals $\Delta V_{th\_nbti}$ plus $V_{th\_design}$. Here, $V_{th\_design}$ is given by technology and $\Delta V_{th\_nbti}$ can be simplified as Equation~(\ref{eq:cor2}) because $V_{th\_offset}$ is set to 0. In Equation~(\ref{eq:conv2}), since $V_{th\_design}$ is known and $\Delta V_{th\_nbti}$ can be derived without unknown $S_{v}$, the convergent value of $V_{th}$ can be obtained.
\begin{equation}
	\centering
	\Delta V_{th\_nbti} = A \cdot a^n \cdot t^n
	\label{eq:cor2}
\end{equation}
\begin{equation}
	\centering
	V_{th} = \Delta V_{th\_nbti} + 0 + V_{th\_design}
	\label{eq:conv2}
\end{equation}
\paragraph{Obtain the value of $S_{v}$ with specific $\Delta V_{th\_offset}$}
Given a specific value of $\Delta V_{th\_offset}$, our objective is to derive corresponding $S_{v}$ value. Since the convergent $V_{th}$ value is obtained in the last step and $V_{th\_design}$ is known, we can derive the corresponding $\Delta V_{th\_nbti}$ using Equation~(\ref{eq:conv}), such that the corresponding $S_{v}$ can be obtained in Equation~(\ref{eq:cor}). 

%Shortly speaking, given a specific value of $\Delta V_{th\_offset}$, our objective is to derive corresponding $S_{v}$ value. We first assume threshold voltages are convergent over a long-term period; then, since all the coefficients of Equation~(\ref{eq:cor}) are known, the value of $\Delta V_{th\_nbti}$ can be obtained.
So far, the conversion from a given specific $\Delta V_{th\_offset}$ to corresponding $\Delta V_{th\_nbti}$ has been constructed. Then, $\Delta V_{th\_nbti}$ must be transformed to aging-induced delay shift. In~\cite{wang2007efficient}, the delay shift is linearly proportional to $\Delta V_{th\_nbti}$:
\begin{equation}
	\centering
	\Delta t_{p\_aged} = C \cdot \Delta V_{th\_nbti}
	\label{eq:vtodelay}
\end{equation}	
where $\Delta t_{p\_aged}$ is BTI-induced delay shift, and $C$ is a constant and fitted to 0.5 after SPICE simulation. Further, the Equation~(\ref{eq:vtodelay}) is modified as following Equation~(\ref{eq:vtodelay2}) to account for the conversion from $\Delta V_{th\_offset}$ to intrinsic delay shift. 
\begin{equation}
	\centering
	\Delta t_{p\_intrinsic} = C \cdot \Delta V_{th\_offset}
	\label{eq:vtodelay2}
\end{equation}	
where $\Delta t_{p\_intrinsic}$ is the delay shift caused by $\Delta V_{th\_offset}$. Up to now, a model is built to convert a given specific $\Delta V_{th\_offset}$ to corresponding $\Delta t_{p\_aged}$ and $\Delta t_{p\_intrinsic}$. As mentioned earlier, $\Delta V_{th\_offset}$ is the fresh offset/gap between high and nominal V\textsubscript{th}. We use the model to derive the intrinsic delay and aging rates of high-V\textsubscript{th} buffer. In our experiments, $\Delta V_{th\_offset}$ is set to a specific value, such that the intrinsic delay of high-V\textsubscript{th} buffers is 1.2X longer than that of nominal buffer, and the aging rates with duty cycle of 20\%, 40\%, 50\%, and 80\% are 0.5\%, 4.1\%, 5.4\%, and 8.2\%, respectively.