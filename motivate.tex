\section{MOTIVATING EXAMPLE}
\label{sec:motivate}
\begin{figure*}[!ht]
    \centering
    \subfigure[Example]{
    	\label{fig:sub:example}
        \includegraphics[width=0.9\columnwidth]{Motivating_example.png}
    }
    \hspace{1.6cm}
    \subfigure[Notations]{
    	\label{fig:sub:notations}
        \includegraphics[width=0.8\columnwidth]{Notations.png}
    }
    \caption{Illustrative example and notations for the proposed framework based on DCC deployment/insertion}
    \label{fig:en}
\end{figure*}

This section introduces an illustrative example which motivates our idea of making aging useful. Consider the circuit in Figure~\ref{fig:sub:example} where \ce{FF_X}, \ce{FF_Y} and \ce{FF_Z} are three edge-triggered flip-flops, and \ce{L_{XY}} and \ce{L_{YZ}} are the corresponding in-between logic networks. Other notations to be used later are listed in Figure~\ref{fig:sub:notations}.

For each pair of flip-flops (e.g., \ce{FF_i} and \ce{FF_j}) between which there exists at least one logic path from \ce{FF_i} to \ce{FF_j}, the following setup-time (Equation (\ref{eq:tsu})) and hold-time (Equation (\ref{eq:th})) constraints need to be satisfied:
\begin{equation}
C_i+T_{cq}+D_{ij}+T_{su}<C_j+T_c
\label{eq:tsu}
\end{equation}
\begin{equation}
C_i+T_{cq}+d_{ij}<C_j+T_h
\label{eq:th}
\end{equation}

Assume that, at year 10, $D_{ij}$ is degraded by 15\%, and both $T_{cq}$ and $T_{su}$ increase by 20\%. By using the predictive model presented in~\cite{wang2010impact},~\cite{wang2007efficient}, we can accurately derive the aging of $C_i$ and $C_j$ due to the regularity and predictability of a typical clock waveform of 50\% duty cycle. In the process technology used (TSMC 45nm GP standard cell series), $C_i$ and $C_j$ are degraded by 13\% under 10-year BTI, i.e., $C_i$ and $C_j$ become 1.13X larger.

To be aging-aware for the example circuit in Figure~\ref{fig:sub:example}, we have to consider aforementioned aging factors in the setup-time constraints on \ce{L_{XY}} and \ce{L_{YZ}}:
\begin{equation}
\mbox{\fontsize{8}{9.6}\selectfont \ce{L_{XY}}:\quad$\textbf{1.13}C_X+1.2T_{cq}+1.15D_{XY}+1.2T_{su}<\textbf{1.13}C_Y+T_c$} 
\label{eq:lxy}
\end{equation}
\begin{equation}
\mbox{\fontsize{8}{9.6}\selectfont \ce{L_{YZ}}:$\quad\textbf{1.13}C_Y+1.2T_{cq}+1.15D_{YZ}+1.2T_{su}<\textbf{1.13}C_Z+T_c$} 
\label{eq:lyz}
\end{equation}
where $C_X$ = $C_Y$ = $C_Z$ = 100, $T_{cq}$ = 8, $T_{su}$ = 2, $D_{XY}$ = 90 and $D_{YZ}$ = 80, as shown in Figure~\ref{fig:sub:example}.
\begin{flushleft}
By re-arranging Equations (\ref{eq:lxy}) and (\ref{eq:lyz}):
\begin{flalign*}
\hspace{1.2em}\ce{L_{XY}}: T_c &> 115.5 &\\
\hspace{1.2em}\ce{L_{YZ}}: T_c &> 104
\end{flalign*}
\end {flushleft}
Therefore, the clock period needs to be larger than 115.5 (dominated by \ce{L_{XY}}) to ensure no setup-time violation over a required lifespan of 10 years. For brevity, we omit the discussion on hold-time constraints, which in our work are actually formulated to ensure no existence of racing due to short paths.

For the objective of minimizing required $T_c$ under aging, we insert one 20\% \textit{duty-cycle converter} (DCC) at the input of buffer 1, and another 80\% DCC at the input of buffer 2. The 20\% (80\%) DCC can decrease (increase) the stress times of downstream clock buffers by converting the clock duty cycle to 20\% (80\%), from a typical duty cycle of 50\%. Therefore, 20\% DCC can mitigate/decelerate the aging of $C_X$ and 80\% DCC can aggravate/accelerate the aging of $C_Y$. In the case of no DCC, $C_X$ and $C_Y$ are degraded by 13\% under 10-year BTI in TSMC 45nm GP standard cell series, while 20\% and 80\% DCCs will degrade $C_X$ and $C_Y$ by 9\% and 16\%, respectively, assuming that the clock paths from the clock source to \ce{FF_X} and \ce{FF_Y} are disjoint.

Consider the new aging factors (due to effects of various clock duty cycles) in the setup-time constraints on \ce{L_{XY}} and \ce{L_{YZ}}:
\begin{equation}
\mbox{\fontsize{8}{9.6}\selectfont \ce{L_{XY}}:\quad$\textbf{1.09}C_X+1.2T_{cq}+1.15D_{XY}+1.2T_{su}<\textbf{1.16}C_Y+T_c$} 
\label{eq:lxy2}
\end{equation}
\begin{equation}
\centering
\mbox{\fontsize{8}{9.6}\selectfont \ce{L_{YZ}}:\quad$\textbf{1.16}C_Y+1.2T_{cq}+1.15D_{YZ}+1.2T_{su}<\textbf{1.16}C_Z+T_c$} 
\label{eq:lyz2}
\end{equation}
By re-arranging Equations (\ref{eq:lxy2}) and (\ref{eq:lyz2}):
\begin{flalign*}
\hspace{1.2em}\ce{L_{XY}}: T_c &> 108.5 &\\
\hspace{1.2em}\ce{L_{YZ}}: T_c &> 104
\end{flalign*}
As it can be seen, we can reduce the required $T_c$ from 115.5 to 108.5 (dominated by \ce{L_{XY}} still), by adding two DCCs in the existing synthesized clock tree to create aging-induced clock skews. The skew for \ce{L_{XY}} (between \ce{FF_X} and \ce{FF_Y}), quantified as 1.16$C_Y$ minus 1.09$C_X$, is useful/beneficial and accounts for the reduction of required $T_c$. A certain level of aging tolerance is thus achieved because aging-induced performance degradation of $D_{XY}$ (plus $T_{cq}$ and $T_{su}$ actually) can be tolerated, by exploring such useful aging-induced clock skews.

One may note that \textit{clock skew scheduling} (CSS)~\cite{fishburn1990clock}, which derives unequal delays for all clock branches prior to \textit{clock tree synthesis} (CTS), can also optimize a circuit for aging tolerance. However, the optimization potential of general CSS is limited since it is difficult to precisely implement a wide range of clock delays during~\cite{li2011optimal}.

In contrast, post-CTS clock skew scheduling based on buffer insertion is another option. We will demonstrate that, if buffer insertion is employed to match our optimization results based on DCC insertion, the number of inserted buffers is usually much larger than the number of inserted DCCs. Also, as described later in Section~\ref{subsec:tpc}, the overhead of a single DCC can be diminished by integrating a DCC with its downstream buffer, which further reveals the cost effectiveness of our proposed DCC-based framework.

\subsection{Aging Prediction Model}
\label{subsec:apm}
Before discussing the proposed framework, we briefly introduce the aging (BTI degradation) model for logic gates/networks~\cite{wang2010impact},~\cite{wang2007efficient} used in our paper. This model enables us to analyze the long-term behavior of BTI-induced MOSFET degradation, with both aging and recovery mechanisms taken into account. First, the degradation of threshold voltage at a given time $t$ can be predicted as:
\begin{equation}
\label{eq:dtv}
\Delta V_{th}=\left(\frac{\sqrt{K_v^2 \cdot T_{clk} \cdot \alpha}}{1-\beta_t^{1/_{2n}}}\right)^{2n}
\end{equation}
where $K_v$ is a function of temperature, electrical field, and carrier concentration, $\alpha$ is the stress probability, and $n$ is the time exponential constant, 0.2 for the used technology. The detailed explanation of each parameter can be found in~\cite{wang2010impact}.

Next, the authors of~\cite{wang2007efficient} simplify this predictive model to be:
\begin{equation}
\label{eq:dtv2}
\Delta V_{th}=b\cdot  \alpha^n \cdot t^n = b \cdot \left(\alpha \cdot t \right)^n
\end{equation}
where $b = 3.9 \times 10^{-3} V \cdot s^{-1/_5}$.

Finally, the rising/falling propagation delay of a gate through the degraded P-type/N-type MOSFET can be derived as a first-order approximation:
\begin{equation}
\label{eq:pd}
\tau_p^\prime = \tau_p + a \cdot \left(\alpha \cdot  t\right)^n
\end{equation}
where $\tau_p$ is the intrinsic delay of the gate without BTI degradation and $a$ is a constant.

We apply Equation (\ref{eq:pd}) to calculate the delay of each gate under BTI, and further estimate the performance of a logic circuit. The coefficient a in Equation (\ref{eq:pd}) for each gate type and each input pin is extracted by fitting HSPICE simulation results in 45nm, Predictive Technology Model (PTM). The simplified long-term model successfully predicts the MOSFET degradation, with less than 5\% loss of accuracy against cycle-by-cycle simulations~\cite{wang2010impact}.
